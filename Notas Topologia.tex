\documentclass{article}
% Language setting
% Replace `english' with e.g. `spanish' to change the document language
\usepackage[spanish]{babel}

% Set page size and margins
% Replace `letterpaper' with`a4paper' for UK/EU standard size
\usepackage[letterpaper,top=2cm,bottom=2cm,left=1cm,right=1cm,marginparwidth=1.75cm]{geometry}

% Useful packages

\usepackage{verbatim}
\usepackage{amsmath}
\usepackage{amsthm}
\theoremstyle{definition}
\newtheorem{definition}{Definición}[section]
\newtheorem{theorem}{Teorema}[section]
\newtheorem{corollary}{Corolario}[theorem]
\newtheorem{lemma}{Lema}[section]
\usepackage{etoolbox}
\usepackage{graphicx}
\usepackage{upgreek}
\newcommand{\bigslant}[2]{{\raisebox{.2em}{$#1$}\left/\raisebox{-.2em}{$#2$}\right.}}
\usepackage{xcolor}
\usepackage[colorlinks=true, allcolors=blue]{hyperref}

\title{Teoremario de Topología}
\author{Wilfredo Gallegos}

\begin{document}
\maketitle

\section{Contenido}
\begin{definition}
 	Sea $X\neq \emptyset$ una clase $\uptau$ de	subconjuntos de X es una \textcolor{blue}{Topología sobre X} si cumple:
 	\begin{enumerate}
 	\item $\emptyset,\ X\in \uptau$
 	\item  La unión de una clase arbitraria de conjuntos en $\uptau$ es un miembro de $\uptau$
 	\item La intersección de una clase finita de miembres de $\uptau$ está en $\uptau$\\
 	Los miembros de $\uptau$ son los \textcolor{blue}{abiertos} de X 
	\end{enumerate}
\end{definition}
\textbf{Nota: } 
\begin{enumerate}
	\item El par $(X,\ \uptau)$ es un \textcolor{blue}{espacio topologíco} 
	\item a los elementos de X se le llaman puntos
\end{enumerate}
\textbf{Nota: }
Un \textcolor{blue}{Espacio Metrizable} es un espacio topológico X con la propiedad que existe una métrica que genera los abiertos de la topología dada.\\

\textbf{prop: }
Si $\uptau_1$ y $\uptau_2$ son topologías sobre X, entonces $\uptau_1\cap\uptau_2$ es topología sobre X
\begin{definition}
	Sea A un subconjunto no vacio del espacio topológico $(X, \uptau)$. Considere la clase 
	\[\uptau_A=\{A\cap G: G\in\uptau \textit{ es abierto de X}\}\]
	a $\uptau_A$ se le llama \textcolor{blue}{topología relativa} sobre A
\end{definition}
\begin{definition}
\hfill
	\begin{enumerate}
		\item Sean X y Y espacios topológicos y f un mapeo entre X y Y. Se dice que f es continua si $f^{-1}(G)$ es un abierto de X para cada abierto G de Y
		\item Se dice que el mapeo es abierto si para cada abierto G de X se cumple que f(G) es abierto de Y
	\end{enumerate}
\end{definition}
\textbf{Nota: } 
Una \textcolor{blue}{Propiedad topológica} es una propiedad que si la tiene el espacio X, la tiene tambien cualquier espacio homeomorfo a X.
\begin{definition}
	Sea $(X,\uptau)$ un esp. top. Un subconjunto $A\subset X$ es \textcolor{blue}{cerrado} ssi $A^c\in\uptau$
\end{definition}
\textbf{VER DEFINICIONES DEL 18/01/23}\\

\begin{definition}
\hfill
	\begin{enumerate}
		\item un punto p de X es interior de $A\subseteq X$ si existe un abierto G $\ni p\in G\subset A$
		\item El interior de A denotado como int(A) o $A^{\circ}$ es el conjunto de todos los puntos interiores de A
	\end{enumerate}
\end{definition}
\begin{definition}
	Un punto frontera de $A\subset X$ es un punto tal que cada vecindad del punto intersecta a A y a $A^c$
\end{definition}
\begin{definition}
	Una base abierta para el espacio topologico $(X,\uptau)$ es una clase de abiertos de X talque cada abierto en $\uptau$ puede escribirse como uniones de miembros de la clase
\end{definition}


\begin{theorem}
	Los enunciados siguientes so nequivalentes
	\begin{enumerate}
		\item Una familia $\beta$ de subconjuntos abiertos del espacio topológico $(X,\ \uptau)$ es una base para $\uptau$ si cada abierto de $\uptau$ es unión de miembros de $\beta$
		\item $\beta\subset\uptau$ es una base para $\uptau$, ssi $\forall G\in \uptau,\ \forall p\in G \exists B_p\in\beta\ni p\in B_p\subset G$ 
	\end{enumerate}
\end{theorem}
\begin{theorem}
	 Sea $\beta$ una familia de subconjuntos de un conjunto no vacio X. Entonces $\beta$ es una clase para una topología $\uptau$ sobre X ssi se cumplen 
	 \begin{enumerate}
	 \item X=$\cup_{b\in\beta} B$
	 \item $\forall B, B^*\in\beta$ se tiene que B$\cap B^*$ la unión de miembros de $\beta \Leftrightarrow$ si p$\in B\cap B^*\exists B_p\in\beta\ni p\in B_p\subset B\cap B^*$ 
	 \end{enumerate}
\end{theorem}
\begin{definition}
	Sea $(X,\uptau)$ un esp. top. Una subclase S de abiertos en $\uptau$ es una subbase de la topología $\uptau$ si las intersecciones finitas de miembros de S producen una base para $\uptau$
\end{definition}
\begin{theorem}
	Sea X cualquier conjunto no vacío y sea S una clase arbitriaria de subconjuntos de X, Entonces S puede construirse en la subbase abierta para una topología sobre X en el sentido que las intersecciones finitas de los miembros de S producen una base para dicha topología.
\end{theorem}
\begin{lemma}
Si S es subbase de las topologías $\uptau$ y $\uptau^*$  sobre X $\Rightarrow\uptau = \uptau^*$
\end{lemma}
\begin{theorem}
Sea X un subconjunto no vacio y sea S una clase de subconjuntos de X. La topología $\uptau$ sobre X, generado por S, y la intersección de todoas las topologías sobre X que contienen a S. 
\end{theorem}
\begin{definition}
	Un espacio topológico que tiene una base contable es un \textcolor{blue}{segundo contable}
\end{definition}
\begin{theorem}{\textbf{Lindelof}}
	Sea X un espacio segundo contable si un abierto no-vacio G de X se puede representar como unión de una clase $\{G_1\}$ de abiertos de X $\Rightarrow$ G puede representarse como unión contable de los $G_i$
\end{theorem}
\begin{definition}{\textbf{Espacios de Hausdorff}}
	Un espacio topológico X es de Hausdorff\label{Hausdorff} si dados $x,y\in X, x\neq y$, existen abiertos U,V de X $\ni x\in U,y\in V$ y $U\cap V = \emptyset$
\end{definition}
\begin{theorem}
	Todo espacio métrico es de Hausdorff(\ref{Hausdorff})
\end{theorem}
\begin{definition}{\textbf{Convergencia}}
	Sea ($X_n$) una sucesión en un espacio topológico X, se dice que ($X_n$) convege a un punto y de X, si para cada vecindad U de y $\exists N\in \mathbf{Z}^+\ni$ si n$\geq N\Rightarrow x\in U$ En este caso y es limite de ($X_n$), y se denota $X_n\to y$
\end{definition}	
\begin{theorem}
	Si X es un espacio de Hausdorff(\ref{Hausdorff}), entonces cada sucesión de puntos ($X_n$) en X converga a lo más a un punto de X.
\end{theorem}
\begin{theorem}
	Cada subconjunto finito A$\subset$X en un Hausdorff(\ref{Hausdorff}) es cerrado
\end{theorem}


\textbf{Continuidad}


\begin{definition}
	Sean $(X,\uptau)$ y $(X,\uptau^*)$ espacios top. El mapeo $f:X\to Y$ es \textcolor{blue}{continuo} si para cada $G\in\uptau^*$ se tiene que $f^{-1}(G)\in\uptau$
\end{definition}
\textbf{Propiedad: } 
Sea $f:X\to Y$ un mapeo entre espacios topológicos. Entonces f es continuo ssi $f(\overline{A})\subset \overline{f(A)}\forall A\subseteq X$
\begin{theorem}
	Composición de mapeos continuos es un mapeo continuo
\end{theorem}
\textbf{Propiedad: } 
Sea $\{\uptau_i\}$ una colección de topologías sobre X. Si $f:X\to Y$ es continuo respecto a cada $\uptau_i\Rightarrow$ f es continua respecto a $\uptau=\cap_i\uptau_i$
\textbf{Propiedad: }
Sea $f:(X,\uptau)\to(Y,\uptau ')$ un mapeo continuo. Si $A\subset X\Rightarrow\ f|_A:(A,\uptau_A)\to (Y,\uptau')$ es continua\\
\textbf{Continuidad local}\\
\begin{definition}
	Sean $(X,\uptau)$ un esp. top. y $x\in X$. Un subconjunto $U\subseteq X$ es \textcolor{blue}{vecindad de x } si $\exists V\in \uptau\ni x\in V\subset U$(x es interior de U)
\end{definition}
\begin{definition}
	La coleccion de todas las vecindades de un punto $x\in X$ se llama \textcolor{blue}{sistema de vencindades de x}\\
	\textcolor{red}{Notacion: } $N_x$
\end{definition}
\textbf{Propiedad: }
Nx es cerrado bajo intersecciones y extensiones\\
\textbf{Propiedad: }Sea A un subconjunto del esp. top. $(X,\uptau)\ni\forall x\in A\exists G\in\uptau\ni x\in G\subset A$. Entonces A es un abierto en $\uptau$\\
\textbf{Propiedad: } Un conjunto G es abierto ssi G es vecindad de cada uno de sus puntos\\
\begin{definition}
	Un mapeo $f:X\to Y$ entre esp. top. es continuo en un punto $x\in X$ si para cada U $U\in N_f(x)\exists V\in N_x\ni f(V)\subset U$
\end{definition}
\begin{theorem}
	 un mapeo $f:X\to Y$ entre espacios topológicos es continuo ssi es continuo en cada punto de X.
\end{theorem}
\begin{LARGE}
\textbf{Homeomorfismo}
\end{LARGE}
\begin{definition}
	Los espacios topológicos $(X,\uptau)$ y $(Y,\uptau ')$ son \textcolor{blue}{Homeomorfos} si existe una función $f:X\to Y\ni$(llamada homeomofismo)
	\begin{enumerate}
		\item f es biyectiva
		\item f y f$^{-1}$ son continuas
	\end{enumerate}
\end{definition}
\begin{theorem}
	Sea $\{f_i:X\to(Y_i,\uptau_i)\}$ una colección de mapeos definidos sobre un conjunto no vacío X sobre los espacios topológicos $(Y_i,\uptau_i)$, sea 
	\[S=\cup_i \{f^{-1}(H):H\in\uptau_i\}\]
	y definimos $\uptau$ como la topología sobre X generada por S, entonces:
	\begin{enumerate}
	\item Todas las $f_i$ son continuas con respecto a $\uptau$
	\item Si $\uptau^*$ es la intersección de todas las topologías sobre X con respecto a las cuales las $f_i$ son continuas, entonces $\uptau=\uptau^*$
	\item $\uptau$ es la topología menos fina sobre X tales que las $f_i$ son continuas
	\item S es una subbase para $\uptau$
	\end{enumerate}
\end{theorem}

\begin{LARGE}
	\textbf{Topología producto}
\end{LARGE}
\begin{definition}
	Sea $X_{\alpha}$ in conjunto $\forall \alpha \in I$. El producto cartesiano de las $x_{\alpha}$ es el conjunto:
	\[\prod_{a\in I} x_{\alpha}:=\{x:I\to\cup_{a\in I}X_{\alpha}\ni x(\alpha) \in X_{\alpha}\forall \alpha \in I\}\]
\end{definition}
\begin{definition}
	Se define la k-esima proyección $\pi_k$ como el mapeo
	\[\pi_k: \prod_{\alpha \in I}X_{\alpha}\to x_{\alpha}\]
	\[\pi_k(w\to w_k)\]
\end{definition}
\begin{LARGE}
	\textbf{Compactos}
\end{LARGE}
\begin{definition}
	Sea $(X,\uptau)$ un esp. top. Una clase $\{H_i\}$ de abiertos de X es una \textcolor{blue}{cubierta abierta } de X ssi $\cup_iH_i=X$
\end{definition}
\begin{definition}
	Una subclase de una cubierta abierta de X que tambien es cubierta abierta es una subcubierta de la inicial
\end{definition}
\begin{definition}
	Un espacio compacto es un esp. top. en la que cada cubierta abierta tiene una subcubierta abierta finita
\end{definition}
\begin{theorem}
	Todo subespacio cerrado de un espacio compacto es compacto
\end{theorem}
\begin{theorem}
	Cualquier imagen continua de un espacio compacto es compacto\\
	\includegraphics[scale=.1]{../../../../../../../../Downloads/WhatsApp Image 2023-05-30 at 5.18.30 PM.jpeg} 
\end{theorem}
\begin{theorem}
	Los enunciados siguientes son equivalentes
	\begin{enumerate}
	\item X es un espacio compacto
	\item Para cada clase $\{F_i\}$ de cerrados de X $\ni\cap_iF_i=\emptyset$, se cumple que $\{F_i\}$ contiene una subclase finita $\{F_{i_1},...,F_{i_m}\}\ni F_{i_1}\cap ...\cap F_{i_m}=\emptyset$
\end{enumerate}
En el teorema anterior, la contrapuesta de (2) es: Para toda clase de cerrados de X, $\{F_{i_1}\}\ni$ cada subclase finita tiene interseicción no vacia, entonces $\cap_iF_i\neq\emptyset$ 
\end{theorem}
\begin{definition}
	Dada una clase de conjuntos $C=\{C_i\}$ se dice que C tiene la propiedad de intersección finita(PIF) si para cada $c_{i_k},...,c_{i_k}$ se cumple $\bigcap\limits_{j=i} C_{ij}\neq \emptyset$
\end{definition}
\begin{theorem}
	X es un espacio compacto ssi cada clase de cerrados de X que tiene la Pif tiene intersección no vacía
\end{theorem}

%teorema 16

\begin{theorem}
	Un espacio topológico es compacto si cada cubierta abierta básica tiene una subcubierta finita
\end{theorem}
\begin{theorem}
	Un espacio topológico es compacto si cada cubierta abierta subbásica tiene una subcuierta finita
\end{theorem}

%teorema 18

\begin{theorem}
	Sea X un espacio $T_2$, cualquier punto y un subespacio disjunto y compacto, pueden separarse por abiertos, en el sentido que tienen vecindades disjuntas
	
	AGREGAR FIGURA
\end{theorem}

%teorema 19

\begin{theorem}

	Cada subespacio compacto de un $T_2$ es cerrado
\end{theorem}
\begin{theorem}
	Un mapeo biyectivo y continuo de un espacio compacto en un espacio de Hausdorff(\ref{Hausdorff}) es un homeomorfismo
	
	AGREGAR FIGURA
\end{theorem}
\begin{LARGE}
	\textbf{Separación}
\end{LARGE}
\begin{definition}
	Un espacio a X es $T_1$ si para $x,y\in X, x\neq y$ exosten vecindades G y H$\ni$
	\[x\in G \ y \ y\not\in G\]
	\[y\in H \ y \ x\not\in H\]
	AGREGAR FIGURA
\end{definition}
\begin{theorem}
	Un esapcio topológico es $T_1$ ssi los unitarios son cerrados
\end{theorem}

%teorema 22

\begin{theorem}
	Cada subespacio de un $T_1$ es un $T_1$
\end{theorem}
\begin{definition}
	Un espacio X es \textcolor{blue}{regular} ssi satisface:\\
	Si F es un cerrado de X y p$\in$ X $\ni$ p $\not\in$ F, existen abiertos Gy H $\ni F\subset G$ y $\{p\}\subset H$ y $G\cap H=\emptyset$
\end{definition}
\begin{definition}
	Un esp. top. es $T_3$ si es regular y $T_1$
\end{definition}
\begin{theorem}
	Si X es $T_3\Rightarrow$ X es $T_2$
\end{theorem}
\begin{definition}
	Un espacio X es \textcolor{blue}{normal} si para $F_1$ y $F_2$ cerrados disjuntos de X existen vecindades disjuntas G y H$\ni$
	\[F_1\subset G\ y \ F_2\subset H\]
\end{definition}
\begin{definition}
	Un esp. top que es normal y $T_1$ es un $T_4$
\end{definition}
\begin{theorem}

	Los enunciados siguietnes son equivalentes
	\begin{enumerate}
		\item X es normal
		\item Si H es un superconjunto abierto del cerrado F, entonces existe un abierto G, $\ni$
		\[F\subset G\subset \overline{G}\subset H\]
	\end{enumerate}
\end{theorem}
\textbf{Propiedad: }
Si X es $T_4\Rightarrow$ X es un $T_3$
\begin{theorem}{Metrizaacion de Urysohn}
	Si X es un espacio @do contable, normal y $T_1$, entonces X es \textcolor{blue}{Metrizable}
\end{theorem}
\begin{definition}\textbf{Completamente regular}
	X es completamente regular si dados F un cerrado de X y $p\in X\ni p\not\in F$, entonces $\exists$ una función continua $f:X\to [0,1]\ni$
	\[f(F)=\{0\}\ y \ f(p)= 1\]
\end{definition}
\begin{theorem}
	Sea D el conjunto de fracciones diádicas en [0,1], entonces $\overline{D}=$[0,1]
\end{theorem}
\begin{lemma}{Urysohn}
	Sean $F_1$ y $F_2$ cerrados disjuntos de un espacio normal X, entonces existe la función continua 
	\[f:X\to[0,1]\ni\]
	\[f(F_1)=\{0\}\ y \ f(F_2)=\{1\}\]
\end{lemma}
\begin{theorem}
	Un esapcio topológico e s$T_1$ ssi $\forall x\in X$ la sucesión x,x,x,... converga a x y solo a x
\end{theorem}

%teorema 28

\begin{theorem}
	Un espacio topológico es $T_2$ ssi cada sucesión convergente tiene límite único
\end{theorem}
\begin{LARGE}
	\textbf{Redes}
\end{LARGE}
\begin{definition}
	Sea D un conjunto y $\leq$ una relación binaria definida sobre D que satisface las siguientes condiciones 
	\begin{enumerate}
		\item $\leq$ es reflexiva. x$\leq$x $\forall x\in D$
		\item $\leq$ es transitiva
		\item $\leq$ es dirigida, si $x,y\in D\Rightarrow\exists z\in D\ni x\leq z\ y \ y\leq z$\\
		El par ($D,\leq$) es un conjunto dirigido
	\end{enumerate}
\end{definition}
\begin{definition}
	Una \textcolor{blue}{red} es un conjunto X que es un mapeo
	\[w:D\to X\]
	donde ($D,\leq$) es un conjunto dirigido
\end{definition}
\begin{definition}
	Si $(X,\uptau)$ es un esp. top. y $w:D\to X$ es una red, se dice que w \textcolor{blue}{converge a $x\in X$} si para cada abierto U que contiene a X¿existe $d\in D\ni T_d:\{w(e):d\leq e\in D\}\subset U$
\end{definition}
\textcolor{red}{Notación: }$w\to x$
\begin{theorem}
	Sea $(X,\uptau)$ un espacio topológico y sea $A\subseteq X$, entonces $x\in \overline{A}$ ssi existe una red w en A$\ni\ w\to x$
\end{theorem}
\begin{LARGE}
	\textbf{Subredes}
\end{LARGE}
\textbf{Nota: } Un subconjunto D' de un ocnjunto dirigido D es cofinal so $\forall d\in D\exists e \in D'\ni d\leq e$
\begin{definition}
	Sea $w:D\to X$ y $v:\xi\to X$ redes sobre X donde $(D,\leq),(\xi,\succeq)$ son conjuntos dirigidos. Se dice que v es una subred de w si existe una función $h:\xi\to D\ni$
	\begin{enumerate}
		\item h es monotona i.e. $\alpha\succeq\beta\Rightarrow h(\alpha)\succeq h(\beta)$
		\item h es cofinal
		\item $v(\alpha)=w(h\alpha)\forall a \in \xi$
	\end{enumerate}
\end{definition}
\begin{definition}
	\textbf{Subsucesión } una subsucesión de ($X_n$) es una sucesión de la forma ($X_{n_k}$)
\end{definition}
\newpage`
\begin{LARGE}
	\textbf{Filtros}
\end{LARGE}
\begin{definition}
	Sea X un conjunto. Una colección $\mathcal{F}\subseteq \mathcal{P}(X)$ es un \textcolor{blue}{filtro sobre X} si se satisface:
	\begin{enumerate}
		\item $\emptyset\not\in \mathcal{F}$
		\item Si $A\in \mathcal{F}$ y $A\subseteq B\Rightarrow B\in \mathcal{F}$
		\item Si $A,B\in\mathcal{F}\Rightarrow A\cap B\in \mathcal{F}$
	\end{enumerate}	 
\end{definition}
\textbf{Nota: }
\begin{enumerate}
	\item Cualquier colección $S\subseteq \mathcal{P}(X)$ con la PIF genera un filtro que la contiene
	\item Sea F(x) la colección de todos los filtros sobre X. Sea $\leq$ la relacion de contención. Entonces $(F(X),\leq)$ es un conjunto parcialmente ordenado el cual no puede ser lineal
\end{enumerate}
\begin{theorem}
	Sea $X\neq\emptyset$ y $\mathcal{F}_{\alpha}\in F(x),\ \alpha\in I$. Entonces $\cap_i\mathcal{F}_{\alpha}\in F(x)$
\end{theorem}
\begin{theorem}
	Sea X un conjunto y U(x) ina colección de filtros sobre X. Si para cualquier $\mathcal{F}_1,\mathcal{F}_2\in U(x)$ se tiene que $\mathcal{F}_1\subset\mathcal{F}_2$ o $\mathcal{F}_2\subset\mathcal{F}_1\Rightarrow \cup U(x)$ es filtro
\end{theorem}
\begin{LARGE}
	\textbf{Ultrafiltros}
\end{LARGE}
\begin{definition}
	Sean ($X,\leq$) un conjunto parcialmente ordenado, $a\in X$ y $A\subset X$. Se dice que a es un \textcolor{blue}{elemento maximal} de A, si $a\in A$ y si $a\leq b\forall b\in A\Rightarrow a=b$
\end{definition}
\begin{definition}
	Sean ($X,\leq$) un conjunto parcialmente ordenado y $C\subset X$. Se dice que C es \textcolor{blue}{cadena} en X, si $\forall a,b\in C$ se cumple $a\leq b$ o $b\leq a$
\end{definition}
\begin{lemma}{Zorn}
	So X es un conjunto no vacio y parcialmente ordenado $\ni$ cada cadena en X tiene cota superior, entonces X tiene un elemento maximal
\end{lemma}
\begin{definition}
	Sea $X\neq \emptyset$ una familia $U\subset\mathcal{P}(X)$ es un \textcolor{blue}{ultrafiltro} si se cumple
	\begin{enumerate}
		\item U es filtro
		\item Si $\mathcal{F}$ es un filtro sobre X$\ni U\subseteq\mathcal{F}\Rightarrow U=\mathcal{F}$(U es maximal)
	\end{enumerate}
\end{definition}
\begin{theorem}{\textbf{Tarski}}
	Sea X un conjunto y $\mathcal{F}$ un filtro sobre X. Entonces existe un ultrafiltro U sobre x $\ni \mathcal{F}\subset U$
\end{theorem}
\begin{theorem}
	Sea X un conjunto y U un filtro sobre X. Entonces los enunciados soguietnes son equivalentes:
	\begin{enumerate}
		\item U es ultrafiltro
		\item Par acualquier $E\subset U\ni E\cap F\neq \emptyset,\ \forall F\in U$ se tiene que $E\in U$
		\item Si $E\subset X\Rightarrow E\in U$ o $X-E\in U$
		\item Si $A, B\in$ X y $ A\cap B\in U\Rightarrow A\in U$ o $ B\in U$
	\end{enumerate}
\end{theorem}
\begin{definition}
	Una subcolección $\beta\subset\mathcal{F}$ es una \textcolor{blue}{base del filtro $\mathcal{F}$} si $\forall F\in \mathcal{F} \exists B \in \beta\ni B\subset F$
\end{definition}
\begin{theorem}
	Una familia $\beta$ de subconjuntos no vacios de X es base de algún filtro sobre X ssi $\forall B_1,B_2 \in \beta\exists B_3\in\beta\ni B_3\subset B_1\cap B_2$
\end{theorem}
\begin{theorem}
	Sean X un espacio topológico y $\mathcal{F}$ un filtro sobre X. Entonces a familia $\beta=\{\overline{F}\ni F\in \mathcal{F}\}$ es una base de filtros
\end{theorem}
\textbf{Nota: } Al filtro generado por $\beta$ se le llama $\overline{\mathcal{F}}$
\begin{theorem}
	Sean X, Y espacios topológicos, $\overline{\mathcal{F}}$ un filtro sobre X y un mapeo $f:X\to Y$. Entonces $\beta_{f(\overline{\mathcal{F}})}=\{f(F):F\in \mathcal{F}\}$
\end{theorem}
\begin{theorem}
	Sean X, Y espacios topológicos, $\mathcal{F}$ un filtro sobre Y y un mapeo $f:X\to Y$. Si $\forall F\in \mathcal{F}$ se tiene que $f^{-1}(F)\neq\emptyset$, entonces $\beta=\{f^{-1}(F):F\in \mathcal{F}\}$
\end{theorem}
\begin{theorem}
	Sea $X\neq\emptyset$, $\mathcal{F}$ un filtro sobre X y $E\subset X$. Si $B=\{F\cap E: F\in \mathcal{F}\}$ 
y $B^{'}=\{F\cap (X-E): F\in \mathcal{F}\}$ entonces 
	\begin{enumerate}
		\item Si $F\cap E\neq \emptyset, \forall F\in \mathcal{F}\Rightarrow \beta$ es base de filtro sobre X
		\item Si$\exists F\in \mathcal{F}\ni F\cap E = \emptyset \Rightarrow$ B' es base de filtro sobre X 
	\end{enumerate}
\end{theorem}
\begin{definition}
	Sean ($(X,\uptau)$) un esp. top., $\mathcal{F}$ un filtro sobre X, $x\in X$ entonces:
	\begin{enumerate}
		\item Se dice que $\mathcal{F}$ converge a x, $\mathcal{F}\to x$ si $\forall V\in N(x)\exists F\in \mathcal{F}\ni F\subset V$\textcolor{red}{IMPORTANTE}
		\item se dice que x es un punto de acumulación de $\mathcal{F}$ si $\forall F\in \mathcal{F}$ y $\forall V\in N(x)$ se cumple que $F\cap V\neq \emptyset$\\
		\textcolor{red}{Notación: } $\mathcal{F}>x$
	\end{enumerate}
\end{definition}
\begin{theorem}[Teorema 32]
	Sea X un conjunto y U un filtro sobre X. Entonces los enunciados siguietnes son equivalentes:
	\begin{enumerate}
		\item U es ultrafiltro
		\item Par acualquier $E\subset U\ni E\cap F\neq \emptyset,\ \forall F\in U$ se tiene que $E\in U$
		\item Si $E\subset X\Rightarrow E\in U$ o $X-E\in U$
		\item Si $A, B\in$ X y $ A\cap B\in U\Rightarrow A\in U$ o $ B\in U$
	\end{enumerate}
\end{theorem}
\begin{definition}
	Sea $(X,\uptau)$ un esp. top. $\mathcal{F}$ es filtro sobre X y $x\in X$. Entonces $F\to x$ si $\forall V\in N(x)\exists F\in\mathcal{F}\ni F\subset V$
\end{definition}
\begin{theorem}
	Sean $(X,\uptau)$ un espacio topológico, $\mathcal{F}$ es filtro sobre X y $x\in$X. Entonces $F\to x$ s $\forall V\in N(x)\exists F\in\mathcal{F}\ni F\subset V$
\end{theorem}
\begin{theorem}
	Sea X un espacio topológico. $x\in X$ y $\mathcal{F}$ un filtro sobre $X\ni F\to x$. Si G es un filtro sobre $X\ni F\subset G\Rightarrow G\to x$
\end{theorem}
\begin{definition}
	Se dice que una base de filtro $\beta$ converge a un punto $x\in X$ si $\forall V\in N(x)\exists B\in\beta\ni B\subset V$
	\textbf{Notación: } $\beta\to x$
\end{definition}
\begin{theorem}
	Sean X un espacio topológico, $\mathcal{F}$ un filtro sobre X, $\beta$ una base de filtro para $\mathcal{F}$ y $x\in X$. Entonces $\mathcal{F}\to x$ ssi $\beta \to x$
\end{theorem}
\textbf{Propiedad: }
Sean X un esp. top. $x\in X$ y $\mathcal{F}$ un filtro sobre X$\ni \mathcal{F}\to x$. Si G es un filtro sobre x $\ni \mathcal{F}\subset G\Rightarrow G\to x$
\begin{theorem}
	Sean X un espacio topológico, $x\in X$ y $\mathcal{F}$ un filtro sobre X. Los enunciados siguientes son equivalentes:
	\begin{enumerate}
		\item xes un punto de acumulación de $\mathcal{F}$ i.e. $\mathcal{F} \succ x$
		\item Existe un filtro G en x $\ni\mathcal{F}\subset G$ y $G\to x$
		\item $x\in \overline{F}, \forall F \in \mathcal{F}$. Es decir, $x\in \cap_{F\in\mathcal{F}}\overline{F}$ 
	\end{enumerate}
\end{theorem}
\begin{theorem}
	Sea X un espacio topológico, U un ultrafiltro sobre x y $x\in X$ entonces $U \succ x$ ssi $U\to x$
\end{theorem}
\begin{theorem}
	Sea X un espacio topológico $x\in X$ y $A\subset X$. Entonces, $x\in\overline{A}$ ssi existe un filtro $\mathcal{F}$ sobre X $\ni F\to x$ y $A\in\mathcal{F}$
\end{theorem}
\begin{theorem}
	Sean X,Y espacios topológicos, $x\in X$  y $f:X\to Y$ una función. Entonces f es continua en x ssi la base de filtros $B_{f(N(x))}=\{f(V):V\in N(x)\}$ converge a f(x)
\end{theorem}
\textbf{Propiedad: } 
Sea $(X,\uptau)$ un esp. top. entocnes X es Hausdorff(\ref{Hausdorff}) ssi todo filtro convergente en X converge a un punto único.
\begin{theorem}
	Sea $I\neq\emptyset$ y $\{X_i,\uptau_i\}\ni i\in I$ una colección de espacios topológicos. Sean $(\prod_{i\in I}x_i,\uptau_p)$ el espacio producto y $\mathcal{F}$ un filtro sobre $\prod_{i\in I}x_i$. Entonces $\mathcal{F}\to x$ ssi $\prod_{i}(\mathcal{F})\to \prod_i(x)$ en $(X_i,\uptau_i) \forall i\in I$
\end{theorem}
\begin{theorem}
	Sea X un espacio topológico. Los enunciados siguientes son equivalentes:
	\begin{enumerate}
		\item X es compacto
		\item Toda colección \textbf{A} de conjuntos cerrados no vacios con la PIF, cumple que $\cap A\neq\emptyset$
		\item Para todo filtro $\mathcal{F}$ sobre X $\exists x\in X\ni \mathcal{F} \succ x$
		\item Todo ultrafiltro sobre X converge
	\end{enumerate}
\end{theorem}
\begin{theorem}\textbf{Tikonov}
	Sea $I\neq\emptyset$ y $\{X_i,\uptau_i\}\ni i\in I$ una colección de espacios topológicos. Entonces $(\prod_{i\in I}x_i,\uptau_p)$ es compacto ssi $x_i$ es compacto $\forall i \in I$
\end{theorem}
\begin{lemma}
	Sean X y Y espacios topológicos, $f:X\to Y$ un mapeo y U un ultrafiltro sobre X. Entonces F(U) *Filtro generado por $\beta = \{f(F):F\in U\}$* es un ultrafiltro sobre Y
\end{lemma}
\begin{definition}
	Sea $(Y_{\alpha},\uptau_{\alpha})$ una familia de esp. top. y sea F={$f_{\alpha}: X\to (Y_{\alpha},\uptau_{\alpha}): \alpha\in I$} una familia de mapeos. La \textcolor{blue}{topología inducida} por F sobre X es la topología menos fina o mas debil que hace a las $f_{\alpha)}$ continuas\\
\textbf{Notación: } $\uptau_F$
\end{definition}
\includegraphics[scale=.25]{../../../../../../../../Downloads/WhatsApp Image 2023-06-01 at 12.24.47 PM.jpeg} \\

\begin{LARGE}
	\textbf{Topología Cociente}
\end{LARGE}
\begin{definition}
	Sea $\{(X_{\alpha},\uptau_{\alpha})\}$ una colección de esp. top. Sea Y un conjunto y F=\{$f_{\alpha}:(Y_{\alpha}\to Y$\} una familia de mapeos. La topología mas fuerte que hace a los $f_{\alpha}$ continuas se llama \textcolor{blue}{Topología co-inducida por F}\label{Topologia coinducida}
\end{definition}
\textbf{Nota: } Sea $(X,\uptau)$ un esp. top. Sea Y un conjunto y $f:(X,\uptau)\to Y$ un mapeo. La topología co-inducida por f sobre Y se llama \textcolor{blue}{topología cociente sobre Y}\label{Top. cociente}
\begin{definition}
	Sean X un esp. top. y Y un conjunto $\ni g:X\to Y$ es un mapeo sobreyectivo, entonces $\uptau_y=\{G\subset Y\ni g^{-1}(G)$ es un abierto de X$\}$ es la topología cociente inducida por g para Y.
\end{definition}
\begin{theorem}
	Si X y Y son espacios topológicos y $f:X\to Y$ es un mapeo continuo sobreyectivo y abierto o cerrado, la topología $\uptau$ de Y es la topología Cociente $\uptau_f$
\end{theorem}
\begin{theorem}
	Sea Y un espacio topológico dotado de la topologia cociente, inducida por el mapeo $f:X\to Y$. Entonces un mapeo arbitrario $g:Y\to Z$ es continuo ssi $g\circ f:X\to Z$ es continuo\\
\includegraphics[scale=0.1]{../../../../../../../../Downloads/WhatsApp Image 2023-05-30 at 8.17.33 PM.jpeg} 
\end{theorem}
\textbf{Nota: }\\
\begin{enumerate}
	\item Sea G la partición deu n esp. top. X y sea $\bigslant{X}{G}$ su conjunto cociente o descomposición
	\item La función $p: X\to \bigslant{X}{G}\ni p(x)=[x]$ se llama la función cociente o función canónica 
	\item $\mathcal{U}=\{U\subset\bigslant{X}{G}\ni p^{-1}(U)$ es abierta de X $\}$
\end{enumerate}
\begin{theorem}
\hfill
	\begin{enumerate}
		\item U es una topología para $\bigslant{X}{G}$
		\item Si $\bigslant{X}{G}$ tiene la topología cociente, entonces p es continua. *$p:X\to\bigslant{X}{G}\ni p(x)=[x]$*
		\item Si V es una topología para $\bigslant{X}{G}\ni$ p es continua, entoncres $V\subset U$
		\item Si $\bigslant{X}{G}$ tiene la topología cociente y su $A\subset \bigslant{X}{G}\ni p^{-1}(A)$ es cerrado de X $\Rightarrow$ A es cerrado en $\bigslant{X}{G}$
	\end{enumerate}
\end{theorem}
\textbf{Nota: }
Sean X y Y esp. top. y $f:X\to Y$ un mapeo continuo y sobreyectivo. Definamos
\[G_f=\{f^{-1}(y):y\in Y\}\]
Entonces $G_f$ es una partición de X 
\textbf{Nota: } En la función $p:X\to\bigslant{X}{G_f}$ se tiene que $\forall x\in X$, p(x)=[x]=$f^{-1}(f(x))$
\begin{definition}
	Sea $f:X\to Y$ un mapeo continuo y sobreyectivo\\
	Se define: 
	\[\Phi_f:\bigslant{X}{G_f}\to Y\ni\]
	\[\Phi_f([x]):=f(x)\]
	es decir\\
	\begin{center}
\includegraphics[scale=1]{../../../../../../../../Downloads/mapeo cociente.png}
	\end{center}	
\textbf{Propiedad: } $\Phi_f$ es biyectiva(se cumple: $\Phi_f([x])=f(x)$)
\end{definition}
\begin{lemma}
	$\forall A\subset Y$, $\Phi^{-1}_f(A)=p(f^{-1}(A))$
\end{lemma}
\begin{lemma}
	$\forall B\subset\bigslant{X}{G_f}$, $\Phi_f(B)=f(p^{-1}(B))$
\end{lemma}
\begin{theorem}
	Sean $f:X\to Y$ continua y sobreyectiva y  $\bigslant{X}{G_f}$ dotado de la topología cociente. Entonces $\Phi_f:\bigslant{X}{G_f}\to Y$ es continua
\end{theorem}
\begin{theorem}
	Suponga que $f:X\to Y$ es continua y sobreyectiva. Si f es abierta o cerrada, entonces 
	\[ \bigslant{\Phi}{G_f}\to Y\]
	es un homeomorfismo
\end{theorem}
\begin{theorem}
	Si X es compacto, el espacio Y es Hausdorff y $f:X\to Y$ es continua y sobreyectiva, $\Rightarrow \Phi_f:\bigslant{X}{G_f}\to Y$ es homeomorfismo
\end{theorem}

















\newpage
\section{Teoremas-Lemas-Corolarios}
%teorema 1
\begin{theorem}
	Los enunciados siguientes so nequivalentes
	\begin{enumerate}
		\item Una familia $\beta$ de subconjuntos abiertos del espacio topológico $(X,\ \uptau)$ es una base para $\uptau$ si cada abierto de $\uptau$ es unión de miembros de $\beta$
		\item $\beta\subset\uptau$ es una base para $\uptau$, ssi $\forall G\in \uptau,\ \forall p\in G \exists B_p\in\beta\ni p\in B_p\subset G$ 
	\end{enumerate}
\end{theorem}

%teorema 2

\begin{theorem}
	 Sea $\beta$ una familia de subconjuntos de un conjunto no vacio X. Entonces $\beta$ es una clase para una topología $\uptau$ sobre X ssi se cumplen 
	 \begin{enumerate}
	 \item X=$\cup_{b\in\beta} B$
	 \item $\forall B, B^*\in\beta$ se tiene que B$\cap B^*$ la unión de miembros de $\beta \Leftrightarrow$ si p$\in B\cap B^*\exists B_p\in\beta\ni p\in B_p\subset B\cap B^*$ 
	 \end{enumerate}
\end{theorem}

%teorema 3

\begin{theorem}
	Sea X cualquier conjunto no vacío y sea S una clase arbitriaria de subconjuntos de X, Entonces S puede construirse en la subbase abierta para una topología sobre X en el sentido que las intersecciones finitas de los miembros de S producen una base para dicha topología.
\end{theorem}

%lema 1 

\begin{lemma}
Si S es subbase de las topologías $\uptau$ y $\uptau^*$  sobre X $\Rightarrow\uptau = \uptau^*$
\end{lemma}

%teorema 4

\begin{theorem}
Sea X un subconjunto no vacio y sea S una clase de subconjuntos de X. La topología $\uptau$ sobre X, generado por S, y la intersección de todoas las topologías sobre X que contienen a S. 
\end{theorem}

%teorema 5 lindelof

\begin{theorem}{\textbf{Lindelof}}
	Sea X un espacio segundo contable si un abierto no-vacio G de X se puede representar como unión de una clase $\{G_1\}$ de abiertos de X $\Rightarrow$ G puede representarse como unión contable de los $G_i$
\end{theorem}

%teorema 6

\begin{theorem}
	Todo espacio métrico es de Hausdorff 
\end{theorem}

%teorema 7

\begin{theorem}
	Si X es un espacio de Hausdorff, entonces cada sucesión de puntos ($X_n$) en X converga a lo más a un punto de X.
\end{theorem}

%teorema 8

\begin{theorem}
	Cada subconjunto finito A$\subset$X en un Hausdorff es cerrado
\end{theorem}

%teorema 9

\begin{theorem}
	Composición de mapeos continuos es un mapeo continuo
\end{theorem}

%teorema 10

\begin{theorem}
	 un mapeo $f:X\to Y$ entre espacios topológicos es continuo ssi es continuo en cada punto de X.
\end{theorem}

%teorema 11

\begin{theorem}
	Sea $\{f_i:X\to(Y_i,\uptau_i)\}$ una colección de mapeos definidos sobre un conjunto no vacío X sobre los espacios topológicos $(Y_i,\uptau_i)$, sea 
	\[S=\cup_i \{f^{-1}(H):H\in\uptau_i\}\]
	y definimos $\uptau$ como la topología sobre X generada por S, entonces:
	\begin{enumerate}
	\item Todas las $f_i$ son continuas con respecto a $\uptau$
	\item Si $\uptau^*$ es la intersección de todas las topologías sobre X con respecto a las cuales las $f_i$ son continuas, entonces $\uptau=\uptau^*$
	\item $\uptau$ es la topología menos fina sobre X tales que las $f_i$ son continuas
	\item S es una subbase para $\uptau$
	\end{enumerate}
\end{theorem}


\textbf{Compactos}
%teorema 12

\begin{theorem}
	Todo subespacio cerrado de un espacio compacto es compacto
\end{theorem}

%teorema 13

\begin{theorem}
	Cualquier imagen continua de un espacio compacto es compacto\\
	\includegraphics[scale=.1]{../../../../../../../../Downloads/WhatsApp Image 2023-05-30 at 5.18.30 PM.jpeg} 
\end{theorem}

%teorema 14

\begin{theorem}
	Los enunciados siguientes son equivalentes
	\begin{enumerate}
	\item X es un espacio compacto
	\item Para cada clase $\{F_i\}$ de cerrados de X $\ni\cap_iF_i=\emptyset$, se cumple que $\{F_i\}$ contiene una subclase finita $\{F_{i_1},...,F_{i_m}\}\ni F_{i_1}\cap ...\cap F_{i_m}=\emptyset$
\end{enumerate}
En el teorema anterior, la contrapuesta de (2) es: Para toda clase de cerrados de X, $\{F_{i_1}\}\ni$ cada subclase finita tiene interseicción no vacia, entonces $\cap_iF_i\neq\emptyset$ 
\end{theorem}

%teorema 15

\begin{theorem}
	X es un espacio compacto ssi cada clase de cerrados de X que tiene la Pif tiene intersección no vacía
\end{theorem}

%teorema 16

\begin{theorem}

	Un espacio topológico es compacto si cada cubierta abierta básica tiene una subcubierta finita
\end{theorem}

%teorema 17

\begin{theorem}
	Un espacio topológico es compacto si cada cubierta abierta subbásica tiene una subcuierta finita
\end{theorem}

%teorema 18

\begin{theorem}
	Sea X un espacio $T_2$, cualquier punto y un subespacio disjunto y compacto, pueden separarse por abiertos, en el sentido que tienen vecindades disjuntas
	
	AGREGAR FIGURA
\end{theorem}

%teorema 19

\begin{theorem}

	Cada subespacio compacto de un $T_2$ es cerrado
\end{theorem}

%teorema 20

\begin{theorem}
	Un mapeo biyectivo y continuo de un espacio compacto en un espacio de Hausdorff es un homeomorfismo
	
	AGREGAR FIGURA
\end{theorem}


\textbf{Separación}
%teorema 21
\begin{theorem}
	Un esapcio topológico es $T_1$ ssi los unitarios son cerrados
\end{theorem}

%teorema 22

\begin{theorem}
	Cada subespacio de un $T_1$ es un $T_1$
\end{theorem}

%teorema 23

\begin{theorem}
	Si X es $T_3\Rightarrow$ X es $T_2$
\end{theorem}

%teorema 24

\begin{theorem}

	Los enunciados siguietnes son equivalentes
	\begin{enumerate}
		\item X es normal
		\item Si H es un superconjunto abierto del cerrado F, entonces existe un abierto G, $\ni$
		\[F\subset G\subset \overline{G}\subset H\]
	\end{enumerate}
\end{theorem}

%teorema 25

\begin{theorem}{Metrizaacion de Urysohn}
	Si X es un espacio @do contable, normal y $T_1$, entonces X es \textcolor{blue}{Metrizable}
\end{theorem}

%teorema 26

\begin{theorem}
	Sea D el conjunto de fracciones diádicas en [0,1], entonces $\overline{D}=$[0,1]
\end{theorem}

%lema 2

\begin{lemma}{Urysohn}
	Sean $F_1$ y $F_2$ cerrados disjuntos de un espacio normal X, entonces existe la función continua 
	\[f:X\to[0,1]\ni\]
	\[f(F_1)=\{0\}\ y \ f(F_2)=\{1\}\]
\end{lemma}

%teorema 27

\begin{theorem}
	Un esapcio topológico e s$T_1$ ssi $\forall x\in X$ la sucesión x,x,x,... converga a x y solo a x
\end{theorem}

%teorema 28

\begin{theorem}
	Un espacio topológico es $T_2$ ssi cada sucesión convergente tiene límite único
\end{theorem}
\textbf{Redes}

%teorema 29

\begin{theorem}
	Sea $(Y_i,\uptau_i)$ un espacio topológico y sea $A\subseteq X$, entonces $x\in \overline{A}$ ssi existe una red w en A$\ni\ w\to x$
\end{theorem}
\textbf{Filtros}

%teorema 30

\begin{theorem}
	Sea $X\neq\emptyset$ y $\mathcal{F}_{\alpha}\in F(x),\ \alpha\in I$. Entonces $\cap_i\mathcal{F}_{\alpha}\in F(x)$
\end{theorem}

%teorema 31

\begin{theorem}
	Sea X un conjunto y U(x) ina colección de filtros sobre X. Si para cualquier $\mathcal{F}_1,\mathcal{F}_2\in U(x)$ se tiene que $\mathcal{F}_1\subset\mathcal{F}_2$ o $\mathcal{F}_2\subset\mathcal{F}_1\Rightarrow \cup U(x)$ es filtro
\end{theorem}

%lema 3

\begin{lemma}{Zorn}
	So X es un conjunto no vacio y parcialmente ordenado $\ni$ cada cadena en X tiene cota superior, entonces X tiene un elemento maximal
\end{lemma}

%teorema 32

\begin{theorem}{\textbf{Tarski}}
	Sea X un conjunto y $\mathcal{F}$ un filtro sobre X. Entonces existe un ultrafiltro U sobre x $\ni \mathcal{F}\subset U$
\end{theorem}

%teorema 33

\begin{theorem}
	Sea X un conjunto y U un filtro sobre X. Entonces los enunciados soguietnes son equivalentes:
	\begin{enumerate}
		\item U es ultrafiltro
		\item Par acualquier $E\subset U\ni E\cap F\neq \emptyset,\ \forall F\in U$ se tiene que $E\in U$
		\item Si $E\subset X\Rightarrow E\in U$ o $X-E\in U$
		\item Si $A, B\in$ X y $ A\cap B\in U\Rightarrow A\in U$ o $ B\in U$
	\end{enumerate}
\end{theorem}

%teorema 34

\begin{theorem}
	Una familia $\beta$ de subconjuntos no vacios de X es base de algún filtro sobre X ssi $\forall B_1,B_2 \in \beta\exists B_3\in\beta\ni B_3\subset B_1\cap B_2$
\end{theorem}

%teorema 35

\begin{theorem}
	Sean X un espacio topológico y $\mathcal{F}$ un filtro sobre X. Entonces a familia $\beta=\{\overline{F}\ni F\in \mathcal{F}\}$ es una base de filtros
\end{theorem}

%teorema 36

\begin{theorem}
	Sean X, Y espacios topológicos, $\overline{\mathcal{F}}$ un filtro sobre X y un mapeo $f:X\to Y$. Entonces $\beta_{f(\overline{\mathcal{F}})}=\{f(F):F\in \mathcal{F}\}$
\end{theorem}

%teorema 37

\begin{theorem}
	Sean X, Y espacios topológicos, $\mathcal{F}$ un filtro sobre Y y un mapeo $f:X\to Y$. Si $\forall F\in \mathcal{F}$ se tiene que $f^{-1}(F)\neq\emptyset$, entonces $\beta=\{f^{-1}(F):F\in \mathcal{F}\}$
\end{theorem}

%teorema 38

\begin{theorem}
	Sea $X\neq\emptyset$, $\mathcal{F}$ un filtro sobre X y $E\subset X$. Si $B=\{F\cap E: F\in \mathcal{F}\}$ 
y $B^{'}=\{F\cap (X-E): F\in \mathcal{F}\}$ entonces 
	\begin{enumerate}
		\item Si $F\cap E\neq \emptyset, \forall F\in \mathcal{F}\Rightarrow \beta$ es base de filtro sobre X
		\item Si$\exists F\in \mathcal{F}\ni F\cap E = \emptyset \Rightarrow$ B' es base de filtro sobre X 
	\end{enumerate}
\end{theorem}

%teorema 39

\begin{theorem}[Teorema 32]
	Sea X un conjunto y U un filtro sobre X. Entonces los enunciados siguietnes son equivalentes:
	\begin{enumerate}
		\item U es ultrafiltro
		\item Par acualquier $E\subset U\ni E\cap F\neq \emptyset,\ \forall F\in U$ se tiene que $E\in U$
		\item Si $E\subset X\Rightarrow E\in U$ o $X-E\in U$
		\item Si $A, B\in$ X y $ A\cap B\in U\Rightarrow A\in U$ o $ B\in U$
	\end{enumerate}
\end{theorem}

%teorema 40

\begin{theorem}
	Sean $(X,\uptau)$ un espacio topológico, $\mathcal{F}$ es filtro sobre X y $x\in$X. Entonces $F\to x$ s $\forall V\in N(x)\exists F\in\mathcal{F}\ni F\subset V$
\end{theorem}

%teorema 41

\begin{theorem}
	Sea X un espacio topológico. $x\in X$ y $\mathcal{F}$ un filtro sobre $X\ni F\to x$. Si G es un filtro sobre $X\ni F\subset G\Rightarrow G\to x$
	\textbf{Notación: } $\mathcal{F} \succ x$
\end{theorem}

%teorema 42

\begin{theorem}
	Sean X un espacio topológico, $\mathcal{F}$ un filtro sobre X, $\beta$ una base de filtro para $\mathcal{F}$ y $x\in X$. Entonces $\mathcal{F}\to x$ ssi $\beta \to x$
\end{theorem}

%teorema 43

\begin{theorem}
	Sean X un espacio topológico, $x\in X$ y $\mathcal{F}$ un filtro sobre X. Los enunciados siguientes son equivalentes:
	\begin{enumerate}
		\item xes un punto de acumulación de $\mathcal{F}$ i.e. $\mathcal{F} \succ x$
		\item Existe un filtro G en x $\ni\mathcal{F}\subset G$ y $G\to x$
		\item $x\in \overline{F}, \forall F \in \mathcal{F}$. Es decir, $x\in \cap_{F\in\mathcal{F}}\overline{F}$ 
	\end{enumerate}
\end{theorem}

%teorema 44

\begin{theorem}
	Sea X un espacio topológico, U un ultrafiltro sobre x y $x\in X$ entonces $U \succ x$ ssi $U\to x$
\end{theorem}

%teorema 45

\begin{theorem}
	Sea X un espacio topológico $x\in X$ y $A\subset X$. Entonces, $x\in\overline{A}$ ssi existe un filtro $\mathcal{F}$ sobre X $\ni F\to x$ y $A\in\mathcal{F}$
\end{theorem}

%teorema 46

\begin{theorem}
	Sean X,Y espacios topológicos, $x\in X$  y $f:X\to Y$ una función. Entonces f es continua en x ssi la base de filtros $B_{f(N(x))}=\{f(V):V\in N(x)\}$ converge a f(x)
\end{theorem}

%teorema 47

\begin{theorem}
	Sea $I\neq\emptyset$ y $\{X_i,\uptau_i\}\ni i\in I$ una colección de espacios topológicos. Sean $(\prod_{i\in I}x_i,\uptau_p)$ el espacio producto y $\mathcal{F}$ un filtro sobre $\prod_{i\in I}x_i$. Entonces $\mathcal{F}\to x$ ssi $\prod_{i}(\mathcal{F})\to \prod_i(x)$ en $(X_i,\uptau_i) \forall i\in I$
\end{theorem}

%teorema 48

\begin{theorem}
	Sea X un espacio topológico. Los enunciados siguientes son equivalentes:
	\begin{enumerate}
		\item X es compacto
		\item Toda colección \textbf{A} de conjuntos cerrados no vacios con la PIF, cumple que $\cap A\neq\emptyset$
		\item Para todo filtro $\mathcal{F}$ sobre X $\exists x\in X\ni \mathcal{F} \succ x$
		\item Todo ultrafiltro sobre X converge
	\end{enumerate}
\end{theorem}

%teorema 49

\begin{theorem}\textbf{Tikonov}
	Sea $I\neq\emptyset$ y $\{X_i,\uptau_i\}\ni i\in I$ una colección de espacios topológicos. Entonces $(\prod_{i\in I}x_i,\uptau_p)$ es compacto ssi $x_i$ es compacto $\forall i \in I$
\end{theorem}

%lema 4

\begin{lemma}
	Sean X y Y espacios topológicos, $f:X\to Y$ un mapeo y U un ultrafiltro sobre X. Entonces F(U) *Filtro generado por $\beta = \{f(F):F\in U\}$* es un ultrafiltro sobre Y
\end{lemma}

%teorema 50

\begin{theorem}
	Si X y Y son espacios topológicos y $f:X\to Y$ es un mapeo continuo sobre yectivo y abierto o cerrado, la topología $\uptau$ de Y es la topología Cociente $\uptau_f$
\end{theorem}

%teorema 51

\begin{theorem}
	Sea Y un espacio topológico dotado de la topologia cociente, inducida por el mapeo $f:X\to Y$. Entonces un mapeo arbitrario $g:Y\to Z$ es continuo ssi $g\circ f:X\to Z$ es continuo\\
\includegraphics[scale=0.1]{../../../../../../../../Downloads/WhatsApp Image 2023-05-30 at 8.17.33 PM.jpeg} 
\end{theorem}

%teorema 52

\begin{theorem}
	\begin{enumerate}
		\item U es una topología para $\bigslant{X}{G}$
		\item Si $\bigslant{X}{G}$ tiene la topología cociente, entonces p es continua. *$p:X\to\bigslant{X}{G}\ni p(x)=[x]$*
		\item Si V es una topología para $\bigslant{X}{G}\ni$ p es continua, entoncres $V\subset U$
		\item Si $\bigslant{X}{G}$ tiene la topología cociente y su $A\subset \bigslant{X}{G}\ni p^{-1}(A)$ es cerrado de X $\Rightarrow$ A es cerrado en $\bigslant{X}{G}$
	\end{enumerate}
\end{theorem}


\textbf{Clase virtual 26-05-23}

%lema 5

\begin{lemma}
	$\forall A\subset Y$, $\Phi_f^{-1}(A)=p(f^{-1}(A))$
\end{lemma}

%lema 6

\begin{lemma}
	$\forall B\subset \bigslant{X}{G_f}$, $\Phi_f(B)=f(p^{-1}(B))$
\end{lemma}

%teorema 53

\begin{theorem}
	Sean $f:X\to Y$ continua y sobreyectiva y  $\bigslant{X}{G_f}$ dotado de la topología cociente. Entonces $\Phi_f:\bigslant{X}{G_f}\to Y$ es continua
\end{theorem}

%teorema 54

\begin{theorem}
	Suponga que $f:X\to Y$ es continua y sobreyectiva. Si f es abierta o cerrada, entonces 
	\[ \bigslant{\Phi}{G_f}\to Y\]
	es un homeomorfismo
\end{theorem}

%teorema 55

\begin{theorem}
	Si X es compacto, el espacio Y es Hausdorff y $f:X\to Y$ es continua y sobreyectiva, $\Rightarrow \Phi_f:\bigslant{X}{G_f}\to Y$ es homeomorfismo
\end{theorem}

%teorema 56

\begin{theorem}
	Los enunciados siguientes son equivalentes:
	\begin{enumerate}
		\item X es regular
		\item Si U es un abierto de X y si $x\in U$ entonces existe un abierto V de $x\ni x\in V$ y $\overline{V}\subset U$
		\item Cada $x\in X$ tiene una base de vecindades que consiste de cerrados
	\end{enumerate}
\end{theorem}


\end{document}

