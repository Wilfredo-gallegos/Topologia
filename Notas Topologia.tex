\documentclass{article}
% Language setting
% Replace `english' with e.g. `spanish' to change the document language
\usepackage[spanish]{babel}

% Set page size and margins
% Replace `letterpaper' with`a4paper' for UK/EU standard size
\usepackage[letterpaper,top=2cm,bottom=2cm,left=1cm,right=1cm,marginparwidth=1.75cm]{geometry}

% Useful packages

\usepackage{verbatim}
\usepackage{amsmath}
\usepackage{amsthm}
\theoremstyle{definition}
\newtheorem{definition}{Definición}[section]
\newtheorem{theorem}{Teorema}
\newtheorem{corollary}{Corolario}[theorem]
\newtheorem{lemma}{Lema}
\usepackage{etoolbox}
\usepackage{graphicx}
\usepackage{upgreek}
\usepackage{xcolor}
\usepackage[colorlinks=true, allcolors=blue]{hyperref}

\title{Teoremario de Topología}
\author{Wilfredo Gallegos}

\begin{document}
\maketitle

\section{Contenido}
\begin{definition}
 	Sea $X\neq \emptyset$ una clase $\uptau$ de	subconjuntos de X es una \textcolor{blue}{Topología sobre X} si cumple:
 	\begin{enumerate}
 	\item $\emptyset,\ X\in \uptau$
 	\item  La unión de una clase arbitraria de conjuntos en $\uptau$ es un miembro de $\uptau$
 	\item La intersección de una clase finita de miembres de $\uptau$ está en $\uptau$\\
 	Los miembros de $\uptau$ son los \textcolor{blue}{abiertos} de X 
	\end{enumerate}
\end{definition}
\textbf{Nota: } 
\begin{enumerate}
	\item El par $(X,\ \uptau)$ es un \textcolor{blue}{espacio topologíco} 
	\item a los elementos de X se le llaman puntos
\end{enumerate}


\textbf{Nota: } Recordar: X es $T_1$ ssi $\forall x,y\in X,\ x\neq y\exists$ abiertos U y V $\ni$
	\[x\in U \ y \ x\not\in V\]
	\[y\not\in U \ y \ y\in V\]




\section{Teoremas-Lemas-Corolarios}
\begin{theorem}
	Los enunciados siguientes so nequivalentes
	\begin{enumerate}
		\item Una familia $\beta$ de subconjuntos abiertos del espacio topológico $(X,\ \uptau)$ es una base para $\uptau$ si cada abierto de $\uptau$ es unión de miembros de $\beta$
		\item $\beta\subset\uptau$ es una base para $\uptau$, ssi $\forall G\in \uptau,\ \forall p\in G \exists B_p\in\beta\ni p\in B_p\subset G$ 
	\end{enumerate}
\end{theorem}
\begin{theorem}
	 Sea $\beta$ una familia de subconjuntos de un conjunto no vacio X. Entonces $\beta$ es una clase para una topología $\uptau$ sobre X ssi se cumplen 
	 \begin{enumerate}
	 \item X=$\cup_{b\in\beta} B$
	 \item $\forall B, B^*\in\beta$ se tiene que B$\cap B^*$ la unión de miembros de $\beta \Leftrightarrow$ si p$\in B\cap B^*\exists B_p\in\beta\ni p\in B_p\subset B\cap B^*$ 
	 \end{enumerate}
\end{theorem}
\begin{theorem}
	Sea X cualquier conjunto no vacío y sea S una clase arbitriaria de subconjuntos de X, Entonces S puede construirse en la subbase abierta para una topología sobre X en el sentido que las intersecciones finitas de los miembros de S producen una base para dicha topología.
\end{theorem}
\begin{lemma}
Si S es subbase de las topologías $\uptau$ y $\uptau^*$  sobre X $\Rightarrow\uptau = \uptau^*$
\end{lemma}
\begin{theorem}
Sea X un subconjunto no vacio y sea S una clase de subconjuntos de X. La topología $\uptau$ sobre X, generado por S, y la intersección de todoas las topologías sobre X que contienen a S. 
\end{theorem}
\begin{theorem}{Lindelof}
	Sea X un espacio segundo contable si un abierto no-vacio G de X se puede representar como unión de una clase $\{G_1\}$ de abiertos de X $\Rightarrow$ G puede representarse como unión contable de los $G_i$
\end{theorem}
\begin{theorem}
	Todo espacio m'etrico es de Hausdorff 
\end{theorem}
\begin{theorem}
	Si X es un espacio de Hausdorff, entonces cada sucesión de puntos ($X_n$) en X converga a lo más a un punto de X.
\end{theorem}
\begin{theorem}
	Cada subconjunto finito A$\subset$X en un Hausdorff es cerrado
\end{theorem}
\begin{theorem}
	Composición de mapeos continuos es un mapeo continuo
\end{theorem}
\begin{theorem}
	 un mapeo $f:X\to Y$ entre espacios topológicos es continuo ssi es continuo en cada punto de X.
\end{theorem}
\begin{theorem}
	Sea $\{f_i:X\to(Y_i,\uptau_i)\}$ una colección de mapeos definidos sobre un conjunto no vacío X sobre los espacios topológicos $(Y_i,\uptau_i)$, sea 
	\[S=\cup_i \{f^{-1}(H):H\in\uptau_i\}\]
	y definimos $\uptau$ como la topología sobre X generada por S, entonces:
	\begin{enumerate}
	\item Todas las $f_i$ son continuas con respecto a $\uptau$
	\item Si $\uptau^*$ es la intersección de todas las topologías sobre X con respecto a las cuales las $f_i$ son continuas, entonces $\uptau=\uptau^*$
	\item $\uptau$ es la topología menos fina sobre X tales que las $f_i$ son continuas
	\item S es una subbase para $\uptau$
	\end{enumerate}
\end{theorem}
\textbf{Compactos}
\begin{theorem}
	Todo subespacio cerrado de un espacio compacto es compacto
\end{theorem}
\begin{theorem}
	Cualquier imagen continua de un espacio compacto es compacto
\end{theorem}
\includegraphics[scale=.1]{../../../../../../../../Downloads/WhatsApp Image 2023-05-30 at 5.18.30 PM.jpeg} 
\begin{theorem}
	Los enunciados siguientes son equivalentes
	\begin{enumerate}
	\item X es un espacio compacto
	\item Para cada clase $\{F_i\}$ de cerrados de X $\ni\cap_iF_i=\emptyset$, se cumple que $\{F_i\}$ contiene una subclase finita $\{F_{i_1},...,F_{i_m}\}\ni F_{i_1}\cap ...\cap F_{i_m}=\emptyset$
\end{enumerate}
En el teorema anterior, la contrapuesta de (2) es: Para toda clase de cerrados de X, $\{F_{i_1}\}\ni$ cada subclase finita tiene interseicci'on no vacia, entonces $\cap_iF_i\neq\emptyset$ 
\end{theorem}
\begin{theorem}
	X es un espacio compacto ssi cada clase de cerrados de X que tiene la Pif tiene intersecci'on no vac'ia
\end{theorem}
\begin{theorem}
	Un espacio topol'ogico es compacto si cada cubierta abierta b'asica tiene una subcubierta finita
\end{theorem}
\begin{theorem}
	Un espacio topol'ogico es compacto si cada cubierta abierta subb'asica tiene una subcuierta finita
\end{theorem}
\begin{theorem}
	Sea X un espacio $T_2$, cualquier punto y un subespacio disjunto y compacto, pueden separarse por abiertos, en el sentido que tienen vecindades disjuntas
	
	AGREGAR FIGURA
\end{theorem}
\begin{theorem}
	Cada subespacio compacto de un $T_2$ es cerrado
\end{theorem}
\begin{theorem}
	Un mapeo biyectivo y continuo de un espacio compacto en un espacio de Hausdorff es un homeomorfismo
	
	AGREGAR FIGURA
\end{theorem}
\textbf{Separaci'on}
\begin{theorem}
	Un esapcio topol'ogico es $T_1$ ssi los unitarios son cerrados
\end{theorem}
\begin{theorem}
	Cada subespacio de un $T_1$ es un $T_1$
\end{theorem}
\begin{theorem}
	Si X es $T_3\Rightarrow$ X es $T_2$
\end{theorem}
\begin{theorem}
	Los enunciados siguietnes son equivalentes
	\begin{enumerate}
		\item X es normal
		\item Si H es un superconjunto abierto del cerrado F, entonces existe un abierto G, $\ni$
		\[F\subset G\subset \overline{G}\subset H\]
	\end{enumerate}
\end{theorem}
\begin{theorem}{Metrizaacion de Urysohn}
	Si X es un espacio @do contable, normal y $T_1$, entonces X es \textcolor{blue}{Metrizable}
\end{theorem}
\begin{theorem}
	Sea D el conjunto de fracciones di'adicas en [0,1], entonces $\overline{D}=$[0,1]
\end{theorem}
\begin{lemma}{Urysohn}
	Sean $F_1$ y $F_2$ cerrados disjuntos de un espacio normal X, entonces existe la funci'on continua 
	\[f:X\to[0,1]\ni\]
	\[f(F_1)-\{0\}\ y \ f(F_2)=\{1\}\]
\end{lemma}
\begin{theorem}
	Un esapcio topol'ogico e s$T_1$ ssi $\forall x\in X$ la sucesi'on x,x,x,... converga a x y solo a x
\end{theorem}
\begin{theorem}
	Un espacio topol'ogico es $T_2$ ssi cada sucesi'on convergente tiene l'imite 'unico
\end{theorem}
\textbf{Redes}
\begin{theorem}
	Sea $(Y_i,\uptau_i)$ un espacio topol'ogico y sea $A\subseteq X$, entonces $x\in \overline{A}$ ssi existe una red w en A$\ni\ w\to x$
\end{theorem}
\textbf{Filtros}
\begin{theorem}
	Sea $X\neq\emptyset$ y $\mathcal{F}_{\alpha}\in F(x),\ \alpha\in I$. Entonces $\cap_i\mathcal{F}_{\alpha}\in F(x)$
\end{theorem}
\begin{theorem}
	Sea X un conjunto y U(x) ina colecci'on de filtros sobre X. Si para cualquier $\mathcal{F}_1,\mathcal{F}_2\in U(x)$ se tiene que $\mathcal{F}_1\subset\mathcal{F}_2$ o $\mathcal{F}_2\subset\mathcal{F}_1\Rightarrow \cup U(x)$ es filtro
\end{theorem}
\begin{lemma}{Zorn}
	So X es un conjunto no vacio y parcialmente ordenado $\ni$ cada cadena en X tiene cota superior, entonces X tiene un elemento maximal
\end{lemma}
\begin{theorem}{Tarski}
	Sea X un conjunto y $\mathcal{F}$ un filtro sobre X. Entonces existe un ultrafiltro U sobre x $\ni \mathcal{F}\subset U$
\end{theorem}
\begin{theorem}
	Sea X un conjunto y U un filtro sobre X. Entonces los enunciados soguietnes son equivalentes:
	\begin{enumerate}
		\item U es ultrafiltro
		\item Par acualquier $E\subset U\ni E\cap F\neq \emptyset,\ \forall F\in U$ se tiene que $E\in U$
		\item Si $E\subset X\Rightarrow E\in U$ o $X-E\in U$
		\item Si $A, B\in$ X y $ A\cap B\in U\Rightarrow A\in U$ o $ B\in U$
	\end{enumerate}
\end{theorem}
\begin{theorem}
	Una familia $\beta$ de subconjuntos no vacios de X es base de alg'un filtro sobre X ssi $\forall B_1,B_2 \in \beta\exists B_3\in\beta\ni B_3\subset B_1\cap B_2$
\end{theorem}
\begin{theorem}
	Sean X un espacio topol'ogico y $\mathcal{F}$ un filtro sobre X. Entonces a familia $\beta=\{\overline{F}\ni F\in \mathcal{F}\}$ es una base de filtros
\end{theorem}
\begin{theorem}
	Sean X, Y espacios topol'ogicos, $\overline{\mathcal{F}}$ un filtro sobre X y un mapeo $f:X\to Y$. Entonces $\beta_{f(\overline{\mathcal{F}})}=\{f(F):F\in \mathcal{F}\}$
\end{theorem}
\begin{theorem}
	Sean X, Y espacios topol'ogicos, $\mathcal{F}$ un filtro sobre Y y un mapeo $f:X\to Y$. Si $\forall F\in \mathcal{F}$ se tiene que $f^{-1}(F)\neq\emptyset$, entonces $\beta=\{f^{-1}(F):F\in \mathcal{F}\}$
\end{theorem}
\begin{theorem}
	Sea $X\neq\emptyset$, $\mathcal{F}$ un filtro sobre X y $E\subset X$. Si $B=\{F\cap E: F\in \mathcal{F}\}$ 
y $B^{'}=\{F\cap (X-E): F\in \mathcal{F}\}$ entonces 
	\begin{enumerate}
		\item Si $F\cap E\neq \emptyset, \forall F\in \mathcal{F}\Rightarrow \beta$ es base de filtro sobre X
		\item Si$\exists F\in \mathcal{F}\ni F\cap E = \emptyset \Rightarrow$ B' es base de filtro sobre X 
	\end{enumerate}
\end{theorem}
\begin{theorem}[Teorema 32]
	Sea X un conjunto y U un filtro sobre X. Entonces los enunciados siguietnes son equivalentes:
	\begin{enumerate}
		\item U es ultrafiltro
		\item Par acualquier $E\subset U\ni E\cap F\neq \emptyset,\ \forall F\in U$ se tiene que $E\in U$
		\item Si $E\subset X\Rightarrow E\in U$ o $X-E\in U$
		\item Si $A, B\in$ X y $ A\cap B\in U\Rightarrow A\in U$ o $ B\in U$
	\end{enumerate}
\end{theorem}
\begin{theorem}
	Sean $(Y_i,\uptau)$ un espacio topol'ogico, $\mathcal{F}$ es filtro sobre X y $x\in$X. Entonces $F\to x$ s $\forall V\in N(x)\exists F\in\mathcal{F}\ni F\subset V$
\end{theorem}
\begin{theorem}
	Sea X un espacio topol'ogico. $x\in X$ y $\mathcal{F}$ un filtro sobre $X\ni F\to x$. Si G es un filtro sobre $X\ni F\subset G\Rightarrow G\to x$
	\textbf{Notaci'on: } $\mathcal{F} \succ x$
\end{theorem}
\begin{theorem}
	Sean X un espacio topol'ogico, $\mathcal{F}$ un filtro sobre X, $\beta$ una base de filtro para $\mathcal{F}$ y $x\in X$. Entonces $\mathcal{F}\to x$ ssi $\beta \to x$
\end{theorem}
\begin{theorem}
	Sean X un espacio topol'ogico, $x\in X$ y $\mathcal{F}$ un filtro sobre X. Los enunciados siguientes son equivalentes:
	\begin{enumerate}
		\item xes un punto de acumulaci'on de $\mathcal{F}$ i.e. $\mathcal{F} \succ x$
		\item Existe un filtro G en x $\ni\mathcal{F}\subset G$ y $G\to x$
		\item $x\in \overline{F}, \forall F \in \mathcal{F}$. Es decir, $x\in \cap_{F\in\mathcal{F}}\overline{F}$ 
	\end{enumerate}
\end{theorem}
\begin{theorem}
	Sea X un espacio topol'ogico, U un ultrafiltro sobre x y $x\in X$ entonces $U \succ x$ ssi $U\to x$
\end{theorem}
\begin{theorem}
	Sea X un espacio topol'ogico $x\in X$ y $A\subset X$. Entonces, $x\in\overline{A}$ ssi existe un filtro $\mathcal{F}$ sobre X $\ni F\to x$ y $A\in\mathcal{F}$
\end{theorem}


\bibliographystyle{alpha}
\bibliography{sample}

\end{document}

