\documentclass{article}
% Language setting
% Replace `english' with e.g. `spanish' to change the document language
\usepackage[spanish]{babel}

% Set page size and margins
% Replace `letterpaper' with`a4paper' for UK/EU standard size
\usepackage[letterpaper,top=2cm,bottom=2cm,left=3cm,right=3cm,marginparwidth=1.75cm]{geometry}

% Useful packages
\newtheorem{definition}{Definición}[section]
\newtheorem{theorem}{Teorema}
\newtheorem{corollary}{Corolario}[theorem]
\newtheorem{lemma}{Lema}
\usepackage{verbatim}
\usepackage{amsmath}
\usepackage{etoolbox}
\usepackage{graphicx}
\usepackage{upgreek}
\usepackage{xcolor}
\usepackage[colorlinks=true, allcolors=blue]{hyperref}

\title{Teoremario de Topología}
\author{Wilfredo Gallegos}

\begin{document}
\maketitle

\section{Contenido}
\begin{definition}
 	Sea $X\neq \emptyset$ una clase $\uptau$ de	subconjuntos de X es una \textcolor{blue}{Topolog'ia sobre X} si cumple:
 	\begin{enumerate}
 	\item $\emptyset,\ X\in \uptau$
 	\item  La uni'on de una clase arbitraria de conjuntos en $\uptau$ es un miembro de $\uptau$
 	\item La intersecci'on de una clase finita de miembres de $\uptau$ est'a en $\uptau$\\
 	Los miembros de $\uptau$ son los \textcolor{blue}{abiertos} de X 
	\end{enumerate}
\end{definition}
\textbf{Nota: } 
\begin{enumerate}
	\item El par $(X,\ \uptau)$ es un \textcolor{blue}{espacio topolog'ico} 
	\item a los elementos de X se le llaman puntos
\end{enumerate}

\section{Teoremas-Lemas-Corolarios}
\begin{theorem}
	Los enunciados siguientes so nequivalentes
	\begin{enumerate}
		\item Una familia $\beta$ de subconjuntos abiertos del espacio topol'ogico $(X,\ \uptau)$ es una base para $\uptau$ si cada abierto de $\uptau$ es uni'on de miembros de $\beta$
		\item $\beta\subset\uptau$ es una base para $\uptau$, ssi $\forall G\in \uptau,\ \forall p\in G \exists B_p\in\beta\ni p\in B_p\subset G$ 
	\end{enumerate}
\end{theorem}
\begin{theorem}
	 Sea $\beta$ una familia de subconjuntos de un conjunto no vacio X. Entonces $\beta$ es una clase para una topolog'ia $\uptau$ sobre X ssi se cumplen 
	 \begin{enumerate}
	 \item X=$\cup_{b\in\beta} B$
	 \item $\forall B, B^*\in\beta$ se tiene que B$\cap B^*$ la uni'on de miembros de $\beta \Leftrightarrow$ si p$\in B\cap B^*\exists B_p\in\beta\ni p\in B_p\subset B\cap B^*$ 
	 \end{enumerate}
\end{theorem}
\begin{theorem}
	Sea X cualquier conjunto no vac'io y sea S una clase arbitriaria de subconjuntos de X, Entonces S puede construirse en la subbase abierta para una topolog'ia sobre X en el sentido que las intersecciones finitas de los miembros de S producen una base para dicha topolog'ia.
\end{theorem}
\begin{lemma}
Si S es subbase de las topolog'ias $\uptau$ y $\uptau^*$  sobre X $\uptau = \uptau^*$
\end{lemma}
\begin{theorem}
Sea X un subconjunto no vacio y sea S una clase de subconjuntos de X. La topolog'ia $\uptau$ sobre X, generado por S, y la intersecci'on de todoas las topolog'ias sobre X que contienen a S. 
\end{theorem}
\begin{theorem}{Lindelof}
	Sea X un espacio segundo contable si un abierto no-vacio G de X se puede representar como uni'on de una clase $\{G_1\}$

\end{theorem}





\bibliographystyle{alpha}
\bibliography{sample}

\end{document}